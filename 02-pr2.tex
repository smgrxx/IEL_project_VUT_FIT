\section{Příklad 2}
\druhyZadani{E}

    \subsection{Výpočet I$_{i}$}
    Sestavime rovnici použitím 2 Kirchhoffova zákona pro výpočet celkového proudu:
    \newline
    -I${\cdot \R{1}}$ - I${\cdot \R{3}}$ - I${\cdot \R{4}}$ - I${\cdot \R{2}}$ + U = 0 
    \newline
    
    I$_{i}$ = $\dfrac{U}{\R{1} + \R{3} + \R{4} + \R{2}}$
    \newline
\begin{figure}[H]
    \centering
    \begin{circuitikz}
		\draw(0,0)to[dcvsource,v^<=U](0,4)
		(0,4)to[R,l^=\R{1}](2,4)
		(2,4)to[R,l^=\R{3}](2,2)
		(2,0)node[circ]{ }
		(2,2)node[circ]{ }
        (2,2)to[R,l^=\R{4}](2,0)
		(0,0)to[R,l^=\R{2}](2,0)
		(3.5,0)node[ocirc]{ }
		(3.5,2)node[ocirc]{ }
		(2,2)--(3.5,2)
		(2,0)--(3.5,0)
		(1,2)node[circulator,rotate=180]{ }
		{[anchor=east](1.8,2.5)node{\I{ }}}
		{[anchor=south](3.5,2)node{A}(3.8,0)node{B}};
        \draw[dashed](3.5,2)--(3.5,0);
	\end{circuitikz}   
    \end{figure} 
\subsection{Výpočet U$_{i}$}

     U$_{i}$ dosadíme namísto R$_{4}$ aby bylo možné počítat podmíněný zdroj napětí a pak používáme 2 Kirhoffův zákon: 
     
    -U$_{i}$ + $I\cdot \R{4} = 0\\$
    U$_{i}$ = I$\cdot \R{4}\\$
\begin{figure}  
    \centering
    \begin{circuitikz}
		\draw(0,0)to[dcvsource,v^<=U](0,4)
		(0,4)to[R,l^=\R{1}](2,4)
		(2,4)to[R,l^=\R{3}](2,2)
		(2,0)node[circ]{ }
		(2,2)node[circ]{ }
            (2,2) to[R,l^=\R{4}] (2,0)
		(3.5,0)node[ocirc]{ }
		(3.5,2)node[ocirc]{ }
		(2,2)--(3.5,2)
		(2,0)--(3.5,0)
		(0,0)to[R,,l^=\R{2}] (2,0)
		(2.7,1.5) node[circulator, rotate=150,scale=0.4]{ }
            {[anchor=east](1.8,2.5)node{\I{ }}}
		{[anchor=south](3.5,2)node{A}(3.8,0)node{B}}
		(1,2)node[circulator,rotate=180]{ };
	    \draw[dashed](3.5,2)to[dcvsource,v^>=U{i}](3.5,0);
	\end{circuitikz}    
    \end{figure}

\subsection{Výpočet R$_{i}$}

    Odpojíme rozhodný odpor, zkratujeme napětový zdroj a označíme si uzly jako uzel A a uzel B namísto uzlů na které byl odpor R$_{5}$ napojen 

    Zjistime odpor R$_{i}$:
\begin{figure}
    \centering
    \setlength{\parindent}{1cm}
    \begin{circuitikz}
		\draw
		(0,0)--
		(0,4)to[R,l^=\R{1}](2,4)
		(2,4)to[R,l^=\R{3}](2,2)
		(2,0)node[circ]{ }
		(2,2)node[circ]{ }
            (2,2)to[R,l^=\R{4}](2,0)
		(0,0)to[R,l^=\R{2}](2,0)
		(3.5,0)node[ocirc]{ }
		(3.5,2) node[ocirc]{ }
		(2,2)--(3.5,2)
		(2,0)--(3.5,0)
		{[anchor=south](3.5,2)node{A}(3.8,0)node{B}};
	\end{circuitikz}
    \end{figure}


    Výpočet R$_{1,2,3}$:
\begin{figure}
    Vypočítame součet všechne rezistoru sériové :

    R$_{1,2,3}$ = $\R{1} + \R{2} + \R{3}\\$
    \newline 

    \centering
        \begin{circuitikz}
		\draw(0,0)--(0,4)--
		(2,4)to[R,l^=\R{123}](2,2)
            (2,2)to[R,l^=\R{4}](2,0)--(0,0)
		(2,2)--(3.5,2) 
		(2,0)--(3.5,0)
            (3.5,0)node[ocirc]{ }
		(3.5,2)node[ocirc]{ }
            (2,0)node[circ]{ }
		(2,2)node[circ]{ }
		{[anchor=south](3.5,2) node{A}(3.8,0)node{B}};
	\end{circuitikz}
\end{figure}

\begin{figure}
    Výpočet R$_{1,2,3,4}:\\$

    Spočítáme paralelne zapojene rezistory:
    \newline

    R$_{1,2,3,4}$ = $\dfrac{\R{1,2,3}\cdot \R{4}}{\R{1,2,3} + \R{4}}$
    \newline

    R$_{i}$ = R$_{1,2,3,4}$

    \centering
    \begin{circuitikz}
		\draw(0,0)to[R,l^=\R{1234}](0,2)--(2,2)
            (0,0)--(2,0)
            (2,2)node[ocirc]{ }
		(2,0)node[ocirc]{ }
		{[anchor=south](2,2)node{A}(2,0)node{B}};
	\end{circuitikz}
    \end{figure}

\begin{figure}
    \subsection{Výpočet I$_{R5}$ a U$_{5}$}

    Použitím náhradneho obvodu počítame I$_{R5}$ a U$_{R5}$ použitím Ohmova zákona:
    \newline
    
    I$_{R5}$ = $\dfrac{U_{i}}{\R{i}+\R{5}}\\$
    \newline
    U$_{5}$ = I$_{5} \cdot \R{5}$
    \newline

    \centering
    \begin{circuitikz}
		\draw(0,0)to[dcvsource,v^<=U{i}](0,2)
		(0,2)to[R,l^=\R{i}](2,2)
            (2,2)to[R,l^=\R{5}](2,0)--(0,0);
	\end{circuitikz}
    \end{figure}


\begin{figure}
    \centering
    \subsection{Dosazení}
    \begin{align*}
        &\I{i} = \dfrac{U}{\R{1} + \R{2} + \R{3} + \R{4}} = \dfrac{250}{150+335+625+245} = \dfrac{250}{1355} = \SI{0.1845}{\ampere} \\
        &\Uu{i} = \I{i} \cdot \R{4} = 0.1845 \cdot 245 = \SI{45.2025}{\volt} \\
        &\R{1,2,3} = \R{1} + \R{2} + \R{3} = 150 + 335 + 625 = \SI{1110}{\ohm} \\
        &\R{1,2,3,4} = \dfrac{R{1,2,3} \cdot \R{4}}{\R{1,2,3} + \R{4}} = \dfrac{1110 \cdot 245}{1110 + 245} = \SI{200.7011}{\ohm} \\   
        &\R{i} = \R{1234} = \SI{200.7011}{\ohm}\\
        &\I{R5} = \dfrac{U{i}}{\R{i}+\R{5}} =\dfrac{45.2025}{200.7011 + 600} = \SI{0.0565}{\ampere} \\
        &\Uu{R5} = \I{R5} \cdot \R{5} = 0.0565 \cdot 600 = \SI {33.8725}{\volt}
    \end{align*}
\end{figure}