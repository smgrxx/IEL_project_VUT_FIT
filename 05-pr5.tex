\section{Příklad 5}
\patyZadani{E}

\subsection{Sestavení diferenciální rovnice}
Sestavíme rovnici pro proud na cívce i\textsubscript{L}$:\\$

$i'_L = \frac{U_L}{L}\\$

Vyjádřime napětí na cívce pomocí druheho Kirchhoffova zákona$:\\$

$U = U_{R} + U_{L}\\$
$U_{L} = U-U_{R}\\$

$i'_{L} = \frac{U-U_{R}}{L} \\$

$i'_{L} = \frac{U-Ri_{L}}{L} \\$

$Li'_{L} + Ri_{L} = U\\$

$30i'_L + 40i_L = 50\\$

Obecný tvar pro cívku$:\\$

$i_L(t)=K(t)\cdot e^{\lambda t}\\$

Najdeme a vypočítame $\lambda$ a $K(t)$$:\\$

\begin{figure}[H]
	$30\lambda + 40 = 0 \\$
	$\lambda = 1 \frac{1}{3}\\$
\end{figure}

Nyní $\lambda$ zobrazena v obecném tvaru, který následně upravíme, a získame defrinciální rovnici cívky$\\$

Upravíme tento tvar$:\\$

$i_L(t) = K(t)\cdot e^{\lambda t}\\$
$i_L(t) = K(t)\cdot e^{-2 t} \\$
$i_L(t)'= K(t)'\cdot e^{-2 t} - 2K(t)e^{-2 t}\\$

Dosadíme to do diferenciální rovnice$:\\$

$(K(t)'\cdot e^{-2 t} - 2K(t) \cdot e^{-2 t}) + 40(K(t)\cdot e^{-2 t}) = 50 \\$
$30K(t)'\cdot e^{-2 t} - 40K(t)\cdot e^{-2 t} + 40K(t)\cdot e^{-2 t} = 50 \\$
$30K(t)'\cdot e^{-2 t} = 50 \\$
$K(t)'\cdot e^{-2 t} = 1 \frac{2}{3} \\$
$K(t)' = 1 \frac{2}{3}\cdot e^{2 t}\\$'

$K(t) = \int 1 \frac{2}{3}\cdot e^{2 t} dt \\$
$K(t) = \frac{5}{6}\cdot e^{2 t} + C\\$

Dosadíme to do analytické rovnice, zkontrolujeme řesení$:\\$

$i_L(t) = K(t) \cdot e^{\lambda t} \\$
$i_L(t) = (\frac{5}{6}\cdot e^{2 t} + C) \cdot e^{-2 t} \\$

$i_L(t)' = \frac{2 \cdot C}{e^{2t}}\\$

Vypočítáme C podle $t = 0$ $:\\$

$i_L(0) = \frac{2 \cdot C}{e^{2\cdot 0}} \\$

$30 = {2 \cdot C}\\$
$C = 15\\$

Konečná rovnice vypada tak$:\\$

$i_L(t)  = \frac{2 \cdot 15}{e^{2t}} \\$