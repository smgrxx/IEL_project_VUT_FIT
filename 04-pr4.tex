\section{Příklad 4}
\ctvrtyZadani{A}

\subsection{Metoda smyčkových proudu}
Definujeme smyčkové proudy v obvodu a vytvoříme 3 rovnici (máme 3 smyčky):
\begin{figure}[H]
    \centering
	\begin{circuitikz}
		\ctikzset{cute inductors} 
		      \draw(0,4)to[sV=\Uu{1}](6,4)
		      (0,4)to[R,l=\R{1}](0,2)
		      (6,4)--(6,2)--(6,0)
                (6,2)node[circ]{ }
                (6,0)to[sV=\Uu{2}](4,0)
		      (4,2)node[circ]{ }
		      (0,2)to[L,l=\Ll{1}](2,2)
                (2,2)to[R,l=\R{2}](4,2)
		      (4,0)node[circ]{ }
		      (4,2)to[L,l=\Ll{2}](6,2)
		      (4,2)to[C=\Cc{1}](4,0)
		      (4,0)to[C=\Cc{2}](0,0)		 
		      (0,2)--(0,0)
		      (3,3)node[circulator,scale=0.6]{ }
		      (2,1)node[circulator,scale=0.6]{ }
		      (5.2,1.1)node[circulator,scale=0.6]{ }
		      {[anchor=south west](3.2,3)node{\I{A}}(2,1.1) node{\I{C}}(5.4,1.1)node{\I{B}}};
	\end{circuitikz}
\end{figure}
Použitím Ohmova zákona (je-li napětí na koncích vodiče stálé, je proud nepřímo úměrný odporu vodiče) a impedance ne lineárních součastek počítáme smyčkové proudy použitím matice:

$\omega = 2\pi f \\$
$\Z{C} = \frac{-j}{\omega C}\\$
$\Z{L} =  j\omega L \\$
\pagebreak

$\I{A}:\Uu{1} + \Uu{L2} + \Uu{R2} + \Uu{L1} + \Uu{R1} = 0\\$
$\I{B}:-\Uu{L2} + \Uu{2} + \Uu{C1} = 0 \\$
$\I{C}:-\Uu{L1} - \Uu{R2} - \Uu{C1} + \Uu{C2} = 0 \\$

$\I{A}:-\I{B}(\Z{L2}) - \I{C}(\Z{R2} + \Z{L1}) = 0\\ $
$\I{B}:-\I{A}(\Z{L2}) + \I{B}(\Z{L2}+\Z{C1}) + \Uu{2} -\I{C}(\Z{C1}) = 0 \\$
$\I{C}:-\I{A}(\Z{R2} + \Z{L1}  )- \I{B}(\Z{C1}) + \I{C}(\Z{L1} +\Z{R2} + \Z{C1} + \Z{C2} ) = 0\\$

\begin{align*}
	\begin{pmatrix}
	   Z_{L2}+Z_{R2}+ Z_{L1}+Z_{R1} &-Z_{L2}&-Z_{R2}-Z_{L1}\\
          -Z_{L2}&Z_{L2}+Z_{C1}&-Z_{C1}\\
          -Z_{R2}-Z_{L1}&-Z_{C1}&Z_{L1}+Z_{R2}+ Z_{C1}+Z_{C2}\\
	\end{pmatrix}
	\cdot
	\begin{pmatrix}
		I_{A}\\
            I_{B}\\ 
            I_{C}\\
	\end{pmatrix}
	   =
	\begin{pmatrix}
	   -U_{1}\\ 
          -U_{2}\\ 
          0
	\end{pmatrix}
\end{align*}

\subsection{Výpočet napětí U$_{C2}$}
Napětí vypočítáme použitím Ohmova zákona a také použijeme modul komplexního čísla (kvůli imaginární části)$:\\$

	$\Uu{C2} = \I{C2}\cdot\Z{C2} = \I{C} \cdot (\frac{-j}{\omega C_2} ) \\$
 
	$|\Uu{L2}| = \sqrt{Re(\Uu{C2})^2 + Im(\Uu{C2})^2}\\$

Fázový posun počítáme ako arktangens (kde x je reálná část imaginárního čísla a y je imaginární část imaginárního čísla)$:\\$

	$\varphi_{L2}= arctan(\frac{Im(\Uu{C2})}{Re(\Uu{C2})}) \times \frac{\pi}{180}\\$

\subsection{Řešení}
	
$\omega = 2\pi f \\$

$\Z{C1} = \frac{-j}{\omega \Cc{1}}\\$

$\Z{C2} = \frac{-j}{\omega \Cc{2}}\\$

$\Z{L1} =  j\omega \Ll{1} \\$

$\Z{L2} =  j\omega \Ll{2}\\$

\begin{align*}
    \begin{pmatrix}
        26 + 96.7611j&-22.6195j &-12 -52.7788j\\
        -439823j&32.6141j&11.3682j\\ 
        -14 - 52.7788j&11.3682j&14 + 19,7569j
    \end{pmatrix}
    \begin{pmatrix}
        I_{A}\\
        I_{B}\\
        I_{C}\\
        \end{pmatrix}
          =
        \begin{pmatrix}
        -3\\
        -5\\
        0\\
        \end{pmatrix}
\end{align*}
	
$\I{A} = (-0.1025 - 0.1024j)\ \si{\ampere} \\$

$\I{B} = (-0.1185 + 0.1245j)\ \si{\ampere} \\$

$\I{C} = (-0.0568 - 0.3126j)\ \si{\ampere}\\$

	
$\I{C2} = \I{C}\\$

$\Uu{C2} = \I{C2} \cdot \Z{C2} = (-0.0568 - 0.3126j) \cdot (-21.6537j) = (-6.7697 + 1.2291j)\ \si{\volt} \\$

$\varphi_{C2}= \arctan(\frac{Im(\Uu{C2})}{Re(\Uu{C2})}) \cdot \frac{\pi}{180}  = (\arctan\frac{1.2291}{-6.7697} + \pi)\cdot\frac{180}{\pi} = 169.7093^\circ \\$

$|\Uu{C2}| = \sqrt{Re(\Uu{C2})^2 + Im(\Uu{C2})^2} = \sqrt{(-6.7697)^2 + 1.2291} = \SI{6.8804}{\volt}$