\section{Příklad 3}
\tretiZadani{A}

\subsection{Nahrazení napětového zdroju}
Nahradíme se napětový zdroj za proudový pro zjednodušení výpočtu vodivostí a následného dosazení do matice

Pro náhradu napěťového zdroje platí: $\I{3} = \Gg{1}U$

\begin{figure}[H]
	\centering
	   \begin{circuitikz}
		\draw(0,0)to[ioosource,i=\I{3}](0,2)
		(0,2)to[R,l=\Gg{1}](2,2)
		(2,2)to[R,l=\Gg{2}](5,2)
		(2,2)to[R,l=\Gg{3}](2,0)--(0,0)
		(4,2)to[R,l=\Gg{4}](4,0)
		(2,0)to[R,l=\Gg{5}](4,0)
		(4,2)--(5,2)to[ioosource,i=\I{2}](5,0)--(4,0)
		(4,0)--(4,-1)
		(2,0)--(2,-1)
		(4,-1)to[ioosource,i=\I{1}](2,-1)
		(4,0)node[circ]{ }
		(4,2)node[circ]{ }
		(2,2)node[circ]{ }
		(2,0)node[circ]{ };
	\end{circuitikz}
	\newline
\end{figure}
Pomocí metody uzlového napětí:
\newline

	$A: \Uu{A}(\Gg{1}+\Gg{2}+\Gg{3}) + \Uu{B}(-\Gg{5}) = -\I{3} \\$
	$B: \Uu{A}(-\Gg{2}) + \Uu{B}(\Gg{2}+\Gg{4}) + \Uu{C}(-\Gg{4}) = -\I{2} \\$
	$C: \Uu{B}(-\Gg{4}) + \Uu{C}(\Gg{4}+\Gg{5}) = \I{2} - \I{1}$
\begin{figure}[H]
    \centering
	\begin{align*}
	   \begin{pmatrix}
		G_{1}+G_{2}+G_{3}&-G_{2}&0\\
		-G_{2}&G_{2}+G_{4}&-G_{4}\\
		0&-G_{4}&G_{4}+G_{5}
	   \end{pmatrix}
            \cdot 
	\begin{pmatrix}
		U_{A}\\
		U_{B}\\
		U_{C}
	\end{pmatrix}
            =
	\begin{pmatrix}
		I_{3}\\
		-I_{2}\\
		I_{2}-I_{1}
	\end{pmatrix}
    \end{align*}
\newline

\subsection{Dosazení}
    \begin{align*}
	\begin{pmatrix}
		0,0547&-0.0204&0\\
		-0.0204&0.0460&-0.0256\\
		0&-0.0256&0,0568
	\end{pmatrix}
            \cdot
	\begin{pmatrix}
		U_{A}\\
		U_{B}\\
		U_{C}
	\end{pmatrix}
            =	
        \begin{pmatrix}
		2.2642\\
		-0.7000 \\
		-0.2000
	\end{pmatrix}
    \end{align*}
    
    \begin{align*}
        \Uu{B} &= \SI{2,0543}{\volt} \\
        \Uu{C} &= \SI{-2.5896}{\volt}
    \end{align*}
    \begin{align*}
        \Uu{R4} &= \Uu{B} - \Uu{C} = 2,0543 - (-2.5896) = \SI{4.6439}{\volt}\\
         \I{R4} &= \frac{\Uu{R4}}{\R{4}} = \frac{4.6439}{39} = \SI{0.1191}{\ampere}
    \end{align*}
\end{figure}